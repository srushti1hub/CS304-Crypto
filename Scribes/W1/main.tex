\documentclass[11pt]{article}
\usepackage[hmargin=1in,vmargin=1in]{geometry}
\usepackage{xcolor}
\usepackage{amsmath,amssymb,amsfonts,url,sectsty,framed,tcolorbox,framed}
\newcommand{\pf}{{\bf Proof: }}
\newtheorem{theorem}{Theorem}
\newtheorem{lemma}{Lemma}
\newtheorem{proposition}{Proposition}
\newtheorem{definition}{Definition}
\newtheorem{remark}{Remark}
\newcommand{\qed}{\hfill \rule{2mm}{2mm}}


\begin{document}
\noindent
\rule{\textwidth}{1pt}
\begin{center}
{\bf [CS304] Introduction to Cryptography and Network Security}
\end{center}
Course Instructor: Dr. Dibyendu Roy \hfill Winter 2022-2023\\
Scribed by: Srushti Rathva (202051183) \hfill Lecture (Week 1)
\\
\rule{\textwidth}{1pt}

%%%%%%%%%%%%%%%%%%%%%%%%%%%%%%%%%%%%%%%%%%%%%%%%%%%%%%%%%%%
%                     INTRODUCTION                        %
%%%%%%%%%%%%%%%%%%%%%%%%%%%%%%%%%%%%%%%%%%%%%%%%%%%%%%%%%%%
\section*{Introduction}
\subsection*{Cryptography}
\textbf{Cryptography} is the practice of using maths algorithms and protocols to \textbf{encod}e and \textbf{decode} messages in order to ensure the \textbf{confidentiality}, \textbf{integrity}, and \textbf{authenticity} of the information being transmitted. \\
In simple words, We \textbf{develop algorithms for security} in Cryptography.

\subsection*{Cryptoanalysis}
\textbf{Cryptoanalysis} is the study of methods for obtaining the \textbf{meaning of encrypted information without access to the secret key.} It involves analyzing and attempting to break cryptographic systems and algorithms in order to gain unauthorized access to the contents of encrypted messages. \\ 
In simple words, We \textbf{try to break security of the designed algorithms} in Cryptoanalysis.

\subsection*{Cryptology}
\textbf{Cryptology} is the science of \textbf{cryptography} and \textbf{cryptoanalysis}. It encompasses both the study of methods for secure communication and the study of methods for breaking those same methods. \\
\textbf{Cryptology = Cryptography + Cryptoanalysis} 
\\ \\
Cryptography provides following services : \\
\textbf{Confidentiality} \\
Protection of personal or sensitive information from being disclosed. \\
\textbf{Integrity} \\
Completeness and accuracy of the data. \\
\textbf{Authentication} \\
Verifying that someone or something is who or what it is claimed. \\
\textbf{Non-repudiation} \\
A mechanism to prove that the sender really sent the message.

%%%%%%%%%%%%%%%%%%%%%%%%%%%%%%%%%%%%%%%%%%%%%%%%%%%%%%%%%%%
%             ENCRYPTION AND DECRYPTION                   %
%%%%%%%%%%%%%%%%%%%%%%%%%%%%%%%%%%%%%%%%%%%%%%%%%%%%%%%%%%%
\section*{Encryption}
\textbf{Encryption} is the process of \textbf{encoding a message or data}. \\
Encryption uses an algorithm and a secret key to transform the original message into a ciphertext that can only be decrypted by someone with the correct key.
\\ \\ 
\textbf{ E(P,E) = C} \\
Where, \\
E = Encryption Function \\
P = Plain Text \\
K = Encyption Key \\
C = Cipher Text \\
\\
For Example, We want to encrypt the message "HELLO" using our custom method. The encrypted message or ciphertext, is obtained by adding or subtracting 5 from each letter of the plaintext.
\begin{align*}
H &\rightarrow M \\
E &\rightarrow J \\
L &\rightarrow Q \\
L &\rightarrow Q \\
O &\rightarrow T
\end{align*}
\\
So the encrypted message is "MJQQT".

\section*{Decryption}
\textbf{Decryption} is the process of \textbf{decoding a ciphertext message back into its original form}. 
\\
It involves using a secret key and the appropriate algorithm to transform the ciphertext back into the original message.
\\ \\ 
\textbf{D(C,K) = P} \\
Where, \\
E = Decryption Function \\
P = Plain Text \\
K = Decyption Key \\
C = Cipher Text \\
\\
For Example, To decrypt the ciphertext message "MJQQT", we can use the same method as the encryption, but this time we will subtract 5 from each letter of the ciphertext.
\begin{align*}
M &\rightarrow H \\
J &\rightarrow E \\
Q &\rightarrow L \\
Q &\rightarrow L \\
T &\rightarrow O
\end{align*}
\\
So the decrypted message is "HELLO".

%%%%%%%%%%%%%%%%%%%%%%%%%%%%%%%%%%%%%%%%%%%%%%%%%%%%%%%%%%%
%                    TYPES OF KEYS                        %
%%%%%%%%%%%%%%%%%%%%%%%%%%%%%%%%%%%%%%%%%%%%%%%%%%%%%%%%%%%
\section*{Types of Keys}
\begin{tabular}{ | l | l | }
\hline
Symmetric Keys & Public Keys  \\
\hline
Only one key i.e secret key is used. & Two keys are used public and secret.\\
\hline
Both encryption and description keys are equal. & Encryption and description keys are not equal. \\ 
$ E_k = D_k $ & $ E_k \neq D_k $. \\
\hline
\end{tabular}

%%%%%%%%%%%%%%%%%%%%%%%%%%%%%%%%%%%%%%%%%%%%%%%%%%%%%%%%%%%
%                     FUNDAMENTALS                        %
%%%%%%%%%%%%%%%%%%%%%%%%%%%%%%%%%%%%%%%%%%%%%%%%%%%%%%%%%%%
\section*{Fundamentals}
\subsection*{One Way Function}
$ f : X \rightarrow Y $ is called one way functionif for given $ x \in X $ it is easy to compute f(x) but for given f(x) it is hard to compute x.
\subsection*{Substitution Box}
S : $ A \rightarrow B $ with $ \lvert A \rvert \leq \lvert B \rvert $ \\
S : \{1,2,3,4\} $ \rightarrow $ \{1,2,3\} 
%%%%%%%%%%%%%%%%%%%%%%%%%%%%%%%%%%%%%%%%%%%%%%%%%%%%%%%%%%%
%                       CIPHERS                           %
%%%%%%%%%%%%%%%%%%%%%%%%%%%%%%%%%%%%%%%%%%%%%%%%%%%%%%%%%%%
\section*{Ciphers}
\subsection*{Caesar Cipher}
The cipher is named after Julius Caesar. It relies on shifting the letters of a message by an agreed number. \\ 
E(x,a) = (x+a)mod26 \\
D(c,a) = (c+26-a)mod26 \\ \\
\textbf{Encryption} \\
let a = 5 \\
Plain text = "ZERO" \\
Z = (25+5)mod26 = 4 = E \\
E = (4+5)mod26 = 9 = J \\
R = (17+5)mod26 = 22 = W \\
O = (14+5)mod26 = 19 = T \\ 
Cipher text = "EJWT" \\ \\
\textbf{Decryption} \\
let a = 5 \\
Cipher text = "EJWT" \\
E = (4+26-5)mod26 = 25 = Z \\
J = (9+26-5)mod26 = 4 = E \\
W = (22+26-5)mod26 = 17 = R \\
T = (19+26-5)mod26 = 14 = O \\ 
Plain text = "ZERO" \\

\subsection*{Transposition Cipher}
The positions of the letters in a message are rearranged to form a new message.\\ \\
\textbf{Encryption and Decryption} \\ \\
e = 
$ \begin{pmatrix}
1 & 2 & 3 & 4 & 5 & 6\\
6 & 4 & 1 & 3 & 5 & 2\\
\end{pmatrix} $
and d = 
$ \begin{pmatrix}
1 & 2 & 3 & 4 & 5 & 6\\
3 & 6 & 4 & 2 & 5 & 1\\
\end{pmatrix} $
\\ \\
Plain text = "CIPHER" \\
Cipher text = "RHCPEI" \\ 

\subsection*{Substitution Cipher}
The positions of the letters in a message are rearranged to form a new message.\\ \\
\textbf{Encryption and Decryption} \\ \\
e = 
$ \begin{pmatrix}
C & I & P & H & E & R \\ 
Q & W & E & R & T & Y \\
\end{pmatrix} $
\\ \\
Plain text = "CIPHER" \\
Cipher text = "QWERTY" \\ 

\subsection*{Affine Cipher}
Each letter in the plaintext is replaced by a letter some fixed number of positions down the alphabet.\\ \\
\textbf{Encryption and Decryption} \\ 
E(x) = (ax + b) mod 26 \\
$ D(x) = a^{-1}(x - b) mod 26 $ \\ 

\subsection*{Playfair Cipher}
It uses a 5x5 matrix of letters, known as the "key square", to encode pairs of letters in the plaintext. \\
The key square is constructed using a keyword, and the remaining letters of the alphabet are added in a row-wise fashion. \\ \\
\textbf{Encryption and Decryption} \\ 
To encode the plaintext, the message is first broken into digraphs (pairs of letters), and then the positions of the letters in the key square are determined. \\
The letters in the digraph are then replaced according to a set of rules based on their positions in the key square. To decode the message, the same process is followed in reverse.

\end{document}
